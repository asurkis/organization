\documentclass{article}

\usepackage[T2A]{fontenc}
\usepackage[utf8]{inputenc}
\usepackage[russian]{babel}
\usepackage{fullpage}
\usepackage{indentfirst}

\usepackage{amsmath}
\usepackage{amsthm}
\usepackage{float}
\usepackage{tikz}

%\usepackage[outputdir=out]{minted}
%\usepackage{algorithmicx}
\usepackage{algpseudocodex}
%\usepackage{algorithm2e}
\usepackage{hyperref}

%\setlength{\parindent}{1.25cm}
%\renewcommand{\baselinestretch}{1.5}
\setlength{\parskip}{6pt}

\begin{document}
    \theoremstyle{definition}
    \newtheorem{definition}{Определение}
    \newtheorem{theorem}{Теорема}
    \newtheorem{statement}{Утверждение}
    \newtheorem{lemma}{Лемма}

    \newcommand{\N}{\mathbb{N}}
    \newcommand{\R}{\mathbb{R}_{>0}}
    \renewcommand{\O}{\mathcal{O}}
    \renewcommand{\o}{o}
    \newcommand{\Const}{\mathit{Const}}

    \renewcommand{\a}{\alpha}
    \renewcommand{\b}{\beta}
    \newcommand{\g}{\gamma}

    \newcommand{\paren}[1]{\left ( #1 \right )}
    \newcommand{\brackets}[1]{\left [ #1 \right ]}
    \newcommand{\braces}[1]{\left \{ #1 \right \}}
    \newcommand{\floor}[1]{\left \lfloor #1 \right \rfloor}
    \newcommand{\ceil}[1]{\left \lceil #1 \right \rceil}
    \newcommand{\abs}[1]{\left | #1 \right |}

    \tableofcontents
    \section{Матрица поворота и кватернионы}
\subsection{Двумерная матрица поворота}
\begin{align*}
    x_1 & = r \cos \alpha \\
    y_1 & = r \sin \alpha \\
    x_2
    & = r \cos (\alpha + \phi) = \\
    & = r \cos \alpha \cos \phi - r \sin \alpha \sin \phi = \\
    & = x_1 \cos \phi - y_1 \sin \phi \\
    y_2
    & = r \sin (\alpha + \phi) = \\
    & = r \sin \alpha \cos \phi + r \cos \alpha \sin \phi = \\
    & = y_1 \cos \phi + x_1 \sin \phi = \\
    R_2 & =
    \begin{bmatrix}
        \cos \phi & -\sin \phi \\
        \sin \phi & \cos \phi \\
    \end{bmatrix} \\
    \begin{bmatrix} x_2 \\ y_2 \end{bmatrix}
    & =
    \begin{bmatrix}
        \cos \phi & -\sin \phi \\
        \sin \phi & \cos \phi \\
    \end{bmatrix} \cdot
    \begin{bmatrix} x_1 \\ y_1 \end{bmatrix}
\end{align*}

\subsection{Трёхмерный базис}
Пусть $(\vec{e}_x, \vec{e}_y, \vec{e}_z)$
--- ортонормированный базис,
т.е. $\vec{e}_z = \vec{e}_x \times \vec{e}_y$.
Тогда
$\vec{r}_e = [
    \vec{e}_x \cdot \vec{r},
    \vec{e}_y \cdot \vec{r},
    \vec{e}_z \cdot \vec{r}]^T$
и
$\vec{r} = \vec{e}_x r_{ex} + \vec{e}_y r_{ex} + \vec{e}_z + r_{ex}$.
То есть

\[
    T =
    \begin{bmatrix}
        e_{xx} & e_{xy} & e_{xz} \\
        e_{yx} & e_{yy} & e_{yz} \\
        e_{zx} & e_{zy} & e_{zz} \\
    \end{bmatrix}
    \quad
    T^{-1} = T^T
\]

\subsection{Трёхмерный поворот}
Известна ось $\vec{omega} = \vec{e}_z$

\[
    R_3 = T^T \cdot R_z \cdot T = \\
\]
\[
    = T^T \cdot
    \begin{bmatrix}
        \cos \phi & -\sin \phi & 0 \\
        \sin \phi &  \cos \phi & 0 \\
        0 & 0 & 1 \\
    \end{bmatrix} \cdot
    \begin{bmatrix}
        e_{xx} & e_{xy} & e_{xz} \\
        e_{yx} & e_{yy} & e_{yz} \\
        e_{zx} & e_{zy} & e_{zz} \\
    \end{bmatrix} =
\]
\[
    = T^T \cdot
    \begin{bmatrix}
        (e_{xx} \cos \phi - e_{yx} \sin \phi) &
        (e_{xy} \cos \phi - e_{yy} \sin \phi) &
        (e_{xz} \cos \phi - e_{yz} \sin \phi) \\
        (e_{xx} \sin \phi + e_{yx} \cos \phi) &
        (e_{xy} \sin \phi + e_{yy} \cos \phi) &
        (e_{xz} \sin \phi + e_{yz} \cos \phi) \\
        e_{zx} & e_{zy} & e_{zz} \\
    \end{bmatrix} =
\]
\[
    =
    \begin{bmatrix}
        e_{xx} & e_{yx} & e_{zx} \\
        e_{xy} & e_{yy} & e_{zy} \\
        e_{xz} & e_{yz} & e_{zz} \\
    \end{bmatrix}
    \cdot
    \begin{bmatrix}
        (e_{xx} \cos \phi - e_{yx} \sin \phi) &
        (e_{xy} \cos \phi - e_{yy} \sin \phi) &
        (e_{xz} \cos \phi - e_{yz} \sin \phi) \\
        (e_{xx} \sin \phi + e_{yx} \cos \phi) &
        (e_{xy} \sin \phi + e_{yy} \cos \phi) &
        (e_{xz} \sin \phi + e_{yz} \cos \phi) \\
        e_{zx} &
        e_{zy} &
        e_{zz} \\
    \end{bmatrix} =
\]
\[
    =
    \begin{bmatrix}
        \begin{bmatrix}
            e_{xx} (e_{xx} \cos \phi - e_{yx} \sin \phi) + e_{yx} (e_{xx} \sin \phi + e_{yx} \cos \phi) + e_{zx}^2 \\
            e_{xx} (e_{xy} \cos \phi - e_{yy} \sin \phi) + e_{yx} (e_{xy} \sin \phi + e_{yy} \cos \phi) + e_{zy}^2 \\
            e_{xx} (e_{xz} \cos \phi - e_{yz} \sin \phi) + e_{yx} (e_{xz} \sin \phi + e_{yz} \cos \phi) + e_{zz}^2 \\
        \end{bmatrix} \\ \\
        \begin{bmatrix}
            e_{xy} (e_{xx} \cos \phi - e_{yx} \sin \phi) + e_{yy} (e_{xx} \sin \phi + e_{yx} \cos \phi) + e_{zx}^2 \\
            e_{xy} (e_{xy} \cos \phi - e_{yy} \sin \phi) + e_{yy} (e_{xy} \sin \phi + e_{yy} \cos \phi) + e_{zy}^2 \\
            e_{xy} (e_{xz} \cos \phi - e_{yz} \sin \phi) + e_{yy} (e_{xz} \sin \phi + e_{yz} \cos \phi) + e_{zz}^2 \\
        \end{bmatrix} \\ \\
        \begin{bmatrix}
            e_{xz} (e_{xx} \cos \phi - e_{yx} \sin \phi) + e_{yz} (e_{xx} \sin \phi + e_{yx} \cos \phi) + e_{zx}^2 \\
            e_{xz} (e_{xy} \cos \phi - e_{yy} \sin \phi) + e_{yz} (e_{xy} \sin \phi + e_{yy} \cos \phi) + e_{zy}^2 \\
            e_{xz} (e_{xz} \cos \phi - e_{yz} \sin \phi) + e_{yz} (e_{xz} \sin \phi + e_{yz} \cos \phi) + e_{zz}^2 \\
        \end{bmatrix} \\
    \end{bmatrix} =
\]
\[
    =
    \begin{bmatrix}
        \begin{bmatrix}
            e_{xx}^2 \cos \phi - e_{xx} e_{yx} \sin \phi + e_{xx} e_{yx} \sin \phi + e_{yx}^2 \cos \phi + e_{zx}^2 \\
            e_{xx} e_{xy} \cos \phi - e_{xx} e_{yy} \sin \phi + e_{xy} e_{yx} \sin \phi + e_{yx} e_{yy} \cos \phi + e_{zy}^2 \\
            e_{xx} e_{xz} \cos \phi - e_{xx} e_{yz} \sin \phi + e_{xz} e_{yx} \sin \phi + e_{yx} e_{yz} \cos \phi + e_{zz}^2 \\
        \end{bmatrix} \\ \\
        \begin{bmatrix}
            e_{xy} e_{xx} \cos \phi - e_{xy} e_{yx} \sin \phi + e_{yy} e_{xx} \sin \phi + e_{yy} e_{yx} \cos \phi + e_{zx}^2 \\
            e_{xy}^2 \cos \phi - e_{xy} e_{yy} \sin \phi + e_{yy} e_{xy} \sin \phi + e_{yy}^2 \cos \phi + e_{zy}^2 \\
            e_{xy} e_{xz} \cos \phi - e_{xy} e_{yz} \sin \phi + e_{yy} e_{xz} \sin \phi + e_{yy} e_{yz} \cos \phi + e_{zz}^2 \\
        \end{bmatrix} \\ \\
        \begin{bmatrix}
            e_{xz} e_{xx} \cos \phi - e_{xz} e_{yx} \sin \phi + e_{yz} e_{xx} \sin \phi + e_{yz} e_{yx} \cos \phi + e_{zx}^2 \\
            e_{xz} e_{xy} \cos \phi - e_{xz} e_{yy} \sin \phi + e_{yz} e_{xy} \sin \phi + e_{yz} e_{yy} \cos \phi + e_{zy}^2 \\
            e_{xz}^2 \cos \phi - e_{xz} e_{yz} \sin \phi + e_{yz} e_{xz} \sin \phi + e_{yz}^2 \cos \phi + e_{zz}^2 \\
        \end{bmatrix} \\
    \end{bmatrix} =
\]
\[
    =
    \begin{bmatrix}
        \begin{bmatrix}
            (e_{xx}^2 + e_{yx}^2) \cos \phi + e_{zx}^2 \\
            (e_{xx} e_{xy} + e_{yx} e_{yy}) \cos \phi + (e_{xy} e_{yx} - e_{xx} e_{yy}) \sin \phi + e_{zy}^2 \\
            (e_{xx} e_{xz} + e_{yx} e_{yz}) \cos \phi + (e_{xz} e_{yx} - e_{xx} e_{yz}) \sin \phi + e_{zz}^2 \\
        \end{bmatrix} \\ \\
        \begin{bmatrix}
            (e_{xy} e_{xx} + e_{yy} e_{yx}) \cos \phi + (e_{yy} e_{xx} - e_{xy} e_{yx}) \sin \phi + e_{zx}^2 \\
            (e_{xy}^2 + e_{yy}^2) \cos \phi + e_{zy}^2 \\
            (e_{xy} e_{xz} + e_{yy} e_{yz}) \cos \phi + (e_{yy} e_{xz} - e_{xy} e_{yz}) \sin \phi + e_{zz}^2 \\
        \end{bmatrix} \\ \\
        \begin{bmatrix}
            (e_{xz} e_{xx} + e_{yz} e_{yx}) \cos \phi + (e_{yz} e_{xx} - e_{xz} e_{yx}) \sin \phi + e_{zx}^2 \\
            (e_{xz} e_{xy} + e_{yz} e_{yy}) \cos \phi + e_{yz} e_{xy} \sin \phi - e_{xz} e_{yy} \sin \phi + e_{zy}^2 \\
            (e_{xz}^2 + e_{yz}^2) \cos \phi + e_{zz}^2 \\
        \end{bmatrix} \\
    \end{bmatrix} =
\]
\[
    =
    \begin{bmatrix}
        \begin{bmatrix}
            (1 - x^2) \cos \phi + x^2 \\
            (e_{xx} e_{xy} + e_{yx} e_{yy}) \cos \phi - z \sin \phi + y^2 \\
            (e_{xx} e_{xz} + e_{yx} e_{yz}) \cos \phi + y \sin \phi + z^2 \\
        \end{bmatrix} \\ \\
        \begin{bmatrix}
            (e_{xy} e_{xx} + e_{yy} e_{yx}) \cos \phi + z \sin \phi + x^2 \\
            (1 - y^2) \cos \phi + y^2 \\
            (e_{xy} e_{xz} + e_{yy} e_{yz}) \cos \phi - x \sin \phi + z^2 \\
        \end{bmatrix} \\ \\
        \begin{bmatrix}
            (e_{xz} e_{xx} + e_{yz} e_{yx}) \cos \phi - y \sin \phi + x^2 \\
            (e_{xz} e_{xy} + e_{yz} e_{yy}) \cos \phi + x \sin \phi + y^2 \\
            (1 - z^2) \cos \phi + z^2 \\
        \end{bmatrix} \\
    \end{bmatrix} =
\]
% sin(2x) = 2 sin x cos x
% cos(2x) = cos^2 x - sin^2 x
% sin(x) =
\[
    =
    \begin{bmatrix}
        \begin{bmatrix}
            (1 - x^2) \cos \phi + x^2 \\
            (e_{xx} e_{xy} + e_{yx} e_{yy}) \cos \phi - z \sin \phi + y^2 \\
            (e_{xx} e_{xz} + e_{yx} e_{yz}) \cos \phi + y \sin \phi + z^2 \\
        \end{bmatrix} \\ \\
        \begin{bmatrix}
            (e_{xy} e_{xx} + e_{yy} e_{yx}) \cos \phi + z \sin \phi + x^2 \\
            (1 - y^2) \cos \phi + y^2 \\
            (e_{xy} e_{xz} + e_{yy} e_{yz}) \cos \phi - x \sin \phi + z^2 \\
        \end{bmatrix} \\ \\
        \begin{bmatrix}
            (e_{xz} e_{xx} + e_{yz} e_{yx}) \cos \phi - y \sin \phi + x^2 \\
            (e_{xz} e_{xy} + e_{yz} e_{yy}) \cos \phi + x \sin \phi + y^2 \\
            (1 - z^2) \cos \phi + z^2 \\
        \end{bmatrix} \\
    \end{bmatrix} =
\]

\end{document}
